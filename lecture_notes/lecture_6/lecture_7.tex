\documentclass{beamer}

\renewcommand\sfdefault{phv}
\renewcommand\familydefault{\sfdefault}
\usetheme{default}
\usepackage{color}
\useoutertheme{default}
\usepackage{texnansi}
\usepackage{color}
\usepackage{marvosym}
\definecolor{bottomcolour}{rgb}{0.32,0.3,0.38}
\definecolor{middlecolour}{rgb}{0.08,0.08,0.16}
\setbeamerfont{title}{size=\Huge}
\setbeamercolor{structure}{fg=gray}
\setbeamertemplate{frametitle}[default]%[center]
\setbeamercolor{normal text}{bg=black, fg=white}
\setbeamertemplate{background canvas}[vertical shading]
[bottom=bottomcolour, middle=middlecolour, top=black]
\setbeamertemplate{items}[circle]
\setbeamerfont{frametitle}{size=\huge}
\setbeamertemplate{navigation symbols}{} %no nav symbols


\usepackage{amsmath,  amsfonts, amsthm, graphicx, subfigure}
%\usepackage{biblatex}
 \usepackage{fancybox, ulem}
 \usepackage{mathtools}
 \usepackage{tabularx}
 \usepackage{tikz}
 \usepackage{movie15}
 %\bibliography{pumping_paper}
\newcommand{\p}{\partial}
\newcommand{\f}{\frac}
\newcommand{\B}{\textbf}
\newcommand{\I}{\textit}
\newcommand{\tab}{\hspace{10mm}}

 % \usetheme{Singapore}
% \usetheme{Warsaw}
  \setbeamertemplate{navigation symbols}{}
\title{Lecture 5}
\author{Austin Baird\\UNC Department of Mathematics\\UNC Department of Biology}
\date{\today} 

\begin{document}
\frame{\titlepage}

\begin{frame}
\frametitle{Summary}
\begin{itemize}

\item Last week we learned: 
\begin{itemize}
\item Runge Kutta.  
\item Second Order v first order method.
\item log log plots.
\end{itemize}
\item Today we will: 
\begin{itemize}
\item Error analysis.
\begin{itemize}
\item Local truncation error.
\item Global error.
\end{itemize}
\item Stability analysis.
\end{itemize}
\end{itemize}

\end{frame}

%--------------------------------------------------------------------------------------------------------------------------------------------------------------------------------------------------------------------------%
\begin{frame}
\frametitle{Introduction}

We want to be able to evaluate the error involved in approximating equations of the form: 

\begin{align*}
\frac{dy}{dt} &= f(t,y)\\
y(t_0) &= y_0
\end{align*}

\ \\

Here $t_0$ is initial time, and $y_0$ is the initial value. How do we analyze our results? 
\end{frame}


%%%--------------------------------------------------------------------------------------------------------------------------------------------------------------------------------------------------------------------------%
\begin{frame}
\frametitle{Approximation with Taylor Series}

We want to approximate $y(t+\Delta t)$ using talyor series. We don't know the solution to the ODE but this will give us insight into our approximation.

\begin{align*}
y(t+\Delta t) &= y(t) + \Delta t \dot{y}(t) +\ldots \Delta t^n y^{(n)}(\tau)\\
t \leq \tau \leq t + \Delta t 
\end{align*}

This is the truncated Taylor series for the function $y(t+\Delta t)$. Note: all of this is analytical and continuous. 

\end{frame}


%%--------------------------------------------------------------------------------------------------------------------------------------------------------------------------------------------------------------------------%
\begin{frame}
\frametitle{Continued}

Now we want to use the taylor series to solve for $\'{y}(t)$ 

\begin{align*}
\dot{y}(t) = \frac{y(t+\Delta t) - y(t)}{\Delta t} + \mathcal{O}(\Delta t)
\end{align*}

Note that there is equality here and that all this is continuous. Now we plug this into our ODE equation and solve for $y(t+\Delta t)$.
\begin{align*} 
y(t+ \Delta t) = y(t) + \Delta t f(t,y(t)) + \mathcal{O}(\Delta t) 
\end{align*}

\end{frame}
%%%--------------------------------------------------------------------------------------------------------------------------------------------------------------------------------------------------------------------------%

\begin{frame}
\frametitle{Local Truncation Error} 

We can now take a difference of the actual solution and our approximation and it becomes clear to see: 

\begin{align*}
\frac{y(t+\Delta t) - y(t)}{\Delta t}  - f(t,y(t)) = \mathcal{O}(\Delta t)
\end{align*}

We can now say that at time $t_{n+1}$ our exact solution to our equation can be written as: 
\begin{align*}
y(t_{n+1}) = y(t_n) + \Delta t f(t_n,y(t_n)) + \Delta t LTE 
\end{align*}

 
\end{frame}
%%%--------------------------------------------------------------------------------------------------------------------------------------------------------------------------------------------------------------------------%

\begin{frame}
\frametitle{Compare to Numerical Approximation} 

We now want to compare this to Euler method (now for the first time we are discussing numerical approximations: $ y(t_n) \approx y_{n}$ 

Recall: 
\begin{align*}
y_{n+1} = y_n + \Delta t f(t_n,y_n)
\end{align*}
\ \\
\textcolor{cyan}{Note:} $ y(t_n) \approx y_{n}$. We can now try and evaluate (global error): 

\begin{align*}
| y(t_{n+1} - y_{n+t} | &=  | y(t_n) - y_n + \Delta t (f(t_n,y(t_n)) - f(t_n,y_n)) + \Delta t LTE \\
&\leq | y(t_n) - y_n | + \Delta t | f(t_n,y(t_n)) - f(t_n,y_n) | + \Delta t LTE \\
&\leq | y(t_n) - y_n | +\Delta t \mathcal{L} | y(t_n) - y_n | + \Delta t LET 
\end{align*}

\end{frame}

%%%%--------------------------------------------------------------------------------------------------------------------------------------------------------------------------------------------------------------------------%


\begin{frame}
\frametitle{Expression for Global Error}

We can now define an expression for global error: \\
\ \\
$e_n = | y(t_n) - y_n |$
\ \\
\begin{align*}
e_{n+1} &= e_n(1+\Delta t \mathcal{L}) + \Delta t LET	 \\
& = e^{\mathcal{L} T } ( \frac{\mathcal{M}}{s\mathcal{L}}\Delta t) = \mathcal{O}(\Delta t)
\end{align*}

This means that forward Euler is a first order global method! But what does this say about the local error? 
\end{frame}
%%%%--------------------------------------------------------------------------------------------------------------------------------------------------------------------------------------------------------------------------%
%%
%%%--------------------------------------------------------------------------------------------------------------------------------------------------------------------------------------------------------------------------%
\begin{frame}
\frametitle{Local Truncation Error}

We can evaluate the error generated by our method in one step as follows, take the difference: 

\begin{align*} 
| y_1 - y(t+\Delta t) | &= ? \\
&= | y_0 + \Delta t f(t_0,y_0) - (y(t_0) + \Delta t \'{y}(t_0) + \mathcal{O}(\Delta t^2)\\
&= \mathcal{O}(\Delta t ^ 2)
\end{align*}


Locally the error is second order, but then what am I defining in my example code as error? 
\end{frame}
%%%--------------------------------------------------------------------------------------------------------------------------------------------------------------------------------------------------------------------------%

\begin{frame}
\frametitle{Norms} 

A norm is a function, $f$ which acts on a vector space $\mathcal{V}$ and has the following properties: For $\textbf{u,v} \epsilon \mathcal{V} $

\begin{align*}
f(a \textbf{u}) &= |a| f(\textbf{u)}\\
f(\textbf{u} + \textbf{v}) &\leq f(\textbf{u}) + f(\textbf{v})\\
f(\textbf{v}) &= 0 \rightarrow \textbf{v} = 0
\end{align*}

Now what I'm having you do is the \textcolor{cyan}{Infinity Norm}:
\begin{align*}
\parallel \textbf{x}\parallel_{inf} = max(|x_1|,|x_2|, \ldots, |x_n|)
\end{align*}

\end{frame}

%
\end{document}

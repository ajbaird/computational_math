\documentclass{beamer}

\renewcommand\sfdefault{phv}
\renewcommand\familydefault{\sfdefault}
\usetheme{default}
\usepackage{color}
\useoutertheme{default}
\usepackage{texnansi}
\usepackage{color}
\usepackage{marvosym}
\definecolor{bottomcolour}{rgb}{0.32,0.3,0.38}
\definecolor{middlecolour}{rgb}{0.08,0.08,0.16}
\setbeamerfont{title}{size=\Huge}
\setbeamercolor{structure}{fg=gray}
\setbeamertemplate{frametitle}[default]%[center]
\setbeamercolor{normal text}{bg=black, fg=white}
\setbeamertemplate{background canvas}[vertical shading]
[bottom=bottomcolour, middle=middlecolour, top=black]
\setbeamertemplate{items}[circle]
\setbeamerfont{frametitle}{size=\huge}
\setbeamertemplate{navigation symbols}{} %no nav symbols


\usepackage{amsmath,  amsfonts, amsthm, graphicx, subfigure}
%\usepackage{biblatex}
 \usepackage{fancybox, ulem}
 \usepackage{mathtools}
 \usepackage{tabularx}
 \usepackage{tikz}
 \usepackage{movie15}
 %\bibliography{pumping_paper}
\newcommand{\p}{\partial}
\newcommand{\f}{\frac}
\newcommand{\B}{\textbf}
\newcommand{\I}{\textit}
\newcommand{\tab}{\hspace{10mm}}

 % \usetheme{Singapore}
% \usetheme{Warsaw}
  \setbeamertemplate{navigation symbols}{}
\title{ Matlab Notes Part 4}
\author{Austin Baird\\UNC Department of Mathematics\\UNC Department of Biology}
\date{\today} 

\begin{document}
\frame{\titlepage}

\begin{frame}
\frametitle{Summary}
\begin{itemize}

\item Last week we learned: 
\begin{itemize}
\item Matlab functions. 
\item Upper triangular matrices.
\item Numerical differentiation.
\item Using built in Matlab functions to solves systems of ODEs. 
\end{itemize}
\item Today we will: 
\begin{itemize}
\item Numerical integration.
\item Rectangle rule and trapezoid rule. 
\item Error of numerical integration, examples of unsolvable integrals. 
\end{itemize}
\end{itemize}

\end{frame}

%--------------------------------------------------------------------------------------------------------------------------------------------------------------------------------------------------------------------------%
\begin{frame}
\frametitle{Problem formulation}

We want to be able to evaluate integrals of the form: 

\begin{align*}
\mathcal{I}(x) = \int^{b}_{a} f(x) dx 
\end{align*}

\ \\
In general this is useful when you can't evaluate the integral given. We want to be able to evaluate this accurately using numerical techniques. 

\end{frame}


%%%--------------------------------------------------------------------------------------------------------------------------------------------------------------------------------------------------------------------------%
\begin{frame}
\frametitle{Idea}

To numerically approximate this we can simply take the area of rectangles under the curve. To do this we subdivide the interval into n partitions: 

\begin{align*}
h&=\frac{b-a}{n}\\
f&\approx f(x_j^{\ast})
\end{align*}

Here $x_j^\ast $ is the midpoint of the jth subinterval. We can then write the area of the jth rectangle as: 

\begin{align*}
A_j = f(x_j^\ast) h 
\end{align*}
Remember the area of a rectangle is just length times height. 
\end{frame}


%%--------------------------------------------------------------------------------------------------------------------------------------------------------------------------------------------------------------------------%
\begin{frame}
\frametitle{Rectangle Rule}

We can now write out the approximation for rectangle rule: 

\begin{align*}
\mathcal{I}(x)= \int^{b}_{a}f(x)ds\approx h[f(x_1^{\ast}) + \hdots + f(x_n^{\ast})]
\end{align*}

for $h=\frac{b-a}{n}$. Lets try and use it! 

\end{frame}
%%%--------------------------------------------------------------------------------------------------------------------------------------------------------------------------------------------------------------------------%

\begin{frame}
\frametitle{Programming Rectangle Rule in Matlab} 

We first need to decretive the x domain: \\

$\gg$ x = a:h:b\\

Then we need to have the integrand: $f(x)$ and evaluate it at the midpoint of each interval: \\
\ \\
$\gg $ for i = 1: size(x), \\
\tab midpoint = $\frac{x(i)+x(i+1)}{2}$\\
\tab sum = sum + f(midpoint)\\
end \\
\ \\

We then want to multiply by h: \\
\ \\

rec = sum*h \\
 \ \\
 Now test this with an integral you know! 
 
\end{frame}
%%%%--------------------------------------------------------------------------------------------------------------------------------------------------------------------------------------------------------------------------%
%
\begin{frame}
\frametitle{Trapezoid Rule}

We now want a more accurate integration method, \textcolor{cyan}{Trapezoid rule :}\\
\ \\
\textcolor{cyan}{Idea:} Take area of trapezoids!! \\ 
\ \\

\begin{align*}
\mathcal{I}(x)= \int^{b}_{a}f(x)ds\approx h[\frac{1}{2}f(a) + f(x_1^{\ast}) + \hdots + f(x_{n-1}^{\ast}) + \frac{1}{2}f(b)]
\end{align*}

Here the area of a trapezoid is used:\\
\ \\
\begin{align*}
\mathcal{A}^{i}_{trap} = \frac{1}{2}[f(x_{i-1}) + f(x_{i})]h
\end{align*} 

This is the area of the ith trapezoid used to approximate the integral. What's different, what's the same? 
\end{frame}
%%%%--------------------------------------------------------------------------------------------------------------------------------------------------------------------------------------------------------------------------%
%%
%%%--------------------------------------------------------------------------------------------------------------------------------------------------------------------------------------------------------------------------%
\begin{frame}
\frametitle{Homework 4}

We want to make a function that computes trapezoid and rectangle rule! There should be \textcolor{cyan}{two functions} (one for each method) and they should: 

\begin{itemize}
\item Take in a function f for the integrand. 
\item Take in the bounds and step size: a, b, and h. 
\end{itemize}

Now that you have these methods, you need to test them a few different ways, have a \textcolor{cyan}{separate script} produce:

\begin{itemize}
\item A figure which graphs the error as a function of step-size (h), for each method. Just choose an integrand (f(x)) that you know the answer to, like $f(x)=x$. Then evaluate this over an interval and compute the error between the actual value and the approximate value. 
\item A figure which plots the difference between Rectangle rule and Trapezoid rule as a function of step size h. 
\end{itemize}

\end{frame}
%%%--------------------------------------------------------------------------------------------------------------------------------------------------------------------------------------------------------------------------%
%
%\begin{frame}
%\frametitle{Homework 3} 
%items to do: 
%
%\begin{itemize}
%\item Alter the functions I gave you so that they represent the lorenz equations.
%\item Make a script which plots the strange attractor: plot(x,z) on the same graph. 
%\item Have this script also plot x, y, z, over the time domain, in the same figure (using subplots) 
%\item Have it make a graph of x for one set of initial conditions = [$x_0 y_0 z_0$] Then on the same figure a plot with these initial conditions: 
%[$(x_0+0.1) (y_0+ 0.1) (z_0 + 0.1)$]. 
%
%\textcolor{cyan}{Note:} Be sure to keep all parameter values greater than zero, other than that you can choose them as you like. Some are more interesting than others...
%\end{itemize}
%\end{frame}
%
%
\end{document}
